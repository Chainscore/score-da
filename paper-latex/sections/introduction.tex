The Blockchain technology is a constantly evolving field that has garnered significant attention and value in recent years. However, despite its growing prominence, the decentralized systems are facing challenges in scaling Blockchains in terms of data availability. Data availability is a critically important aspect of blockchain technology. Data availability refers to the ability of a blockchain network to ensure that all necessary data is accessible and retrievable by all its participants. In a decentralized system, Where multiple nodes work together to validate and store data/transactions, ensuring that the data is avaliable, valid and accessible is quite a important factor for maintaing the integrity and decentralized nature of the network \cite{Avail_Unifying_Blockchain_Network}.


This study provides a Comparative study among the latest avaliable Data availability solutions/models in terms of bandwidth, time and other criteria to provide a bigger overview of pro's and con's of every model.

The Models included in study are as follows:
\begin{itemize}
    \item Polkadot's ELVES
    \item Celestia
    \item Espresso Tiramisu
    \item NEAR
    \item Avail
\end{itemize}

The below is a breif short introduction of the models that will be discussed and analyzed in the study.

\subsection{Polkadot ELVES}
Polkadot is a sharded L0 multichain network the enables cross-chain interoperability and scalablity. It utilizes ELVES (Economic Last Validation Enforcement System) to ensure data availability and validity of the parachains.
Polkadot uses a hybrid consensus mechanism which uses BABE for block production and GRANDPA for its block finality. ELVES uses Reed Solomon encoding to split the data into chunks and then disperse it all across the network.

Polkadot uses Nominated Proof of Stake (NPoS) consensus mechanism which allows token holders to nominate entrusted validators to secure the network.



\subsection{Celestia}
Celestia is one of the first mover in the modular Data Availability solution space playing a major role in L2 and rollups developments. Celestia as of now utilizes 2D-Reed Solomon Erasure coding to split data into chunks and then encode into a matrix extension Namespaced Merkel Trees are then used to ensure the retirval and validity of the data. The root of the NMT is then stored in the block header.

As of now Celestia uses Tenderming consensus mechanism which uses BFT (Byzantine Fault Tolerance) style mechanism to ensure the finality of the blocks

\subsection{Espresso Tiramisu}
Espresso-Tiramisu DA resolves the Data availability scaling issue with a three layred system below is the short overview of these three layers.
\begin{itemize}
    \item Savoiardi (VID Layer) - Erause codes and stores data across all the nodes \cite{Espresso:Hotshot_and_Triamisu}.
    \item Mascarpone (DA Committe Layer) - A Small elected Committee stores the full data and guarantees to efficiently recover data \cite{Espresso:Hotshot_and_Triamisu}.
    \item Cocoa (CDN Layer) - Uploads the data on web2 based CDN solution for seamless and speedy data recovery \cite{Espresso:Hotshot_and_Triamisu}.
\end{itemize}
Espresso utilizes Hotshot consensus which is a optimistically responsive, communication-efficient consensus protocol in a proof-of-stake setting that is resistant to bribing adversaries and scalable to large number of nodes.
\subsection{NEAR}
NEAR provides a high speed DA solution which porvides high transaction volumes with cost-effectivenes.
NEAR utilizes the nightshade sharding mechanism, which parallelizes the network into multiple shards.Each shard processes its own transactions allowing the network to handle a higher volume of transactions approx ~100,000 TPS \cite{NEAR_Nightshade_Scaling_Blockchain}.

As of now NEAR uses sharding-based proof of stake consensus mechanism.NEAR also implements unique validator elections to ensure security and decentralization of the network

\subsection{Avail}
Avail DA helps blockchains scale by providing an abundance of data availability capacity. Its modular design scales data
availability capacity with demand, and transaction data can be cryptographically verified quickly by anyone running an Avail
light client \cite{Avail_Unifying_Blockchain_Network}.Avail utilizes Erasure coding, KZG commitments along with light client to ensure its data availability.

As for consensus Avail uses BABE/GRANDPA hybrid consensus used by polkadot for block production and finality. Avail also provides Application Specific Data Retrieval (ASDR) this helps rollups to fetch and decode their own blobs even tho the block might contain many app's data. \cite{Avail_The_DA_Blockchain}.
\begin{table}[ht]
\centering
\caption{Short Comparison of all DA solutions}
\setlength{\tabcolsep}{4pt} % tighter columns
\renewcommand{\arraystretch}{1.15}
\footnotesize
\begin{tabular}{|p{1.4cm}|p{1.4cm}|p{1.5cm}|p{1.5cm}|p{1.2cm}|p{1.5cm}|}
\hline
\textbf{Feature} &
\makecell[l]{\textbf{Polkadot}\\} &
\textbf{Celestia} &
\makecell[l]{\textbf{Espresso}\\\textbf{Tiramisu}} &
\textbf{NEAR} &
\textbf{Avail} \\
\hline
Consensus & BABE/
GRANDPA & Tendermint & HotShot & Night\-shade & BABE/
GRANDPA \\
\hline
Storage & Erasure Coding  & Reed-Solomon & Three-layer system & Sharding & KZG + Erasure Coding \\
\hline
Block Time & 20s & 15s & \textasciitilde6s & 1s & 20s \\
\hline
Max-Throughput & - & 0.0159 MiB/s & 5 MB/s & -  & 0.2 MiB/s \\
\hline
Block Size & 5 MB & 8 MB & 1 MB & 4 MB & 4 MB \\
\hline
Total Validators & 600 & 100 &  100 & 300 & 1000\\
\hline
Cost/mb & - & 0.08 USD & - & 100kb per NEAR token & 0.0173 USD\\
\hline
Native Token & DOT & - & TIA & NEAR &  AVAIL \\
\hline
Latency & 6-30s & 5-15s & - & 1-2s & 20-40s \\
\hline
TPS & 10 & - & - & 53 & 420 \\
\hline
Txs per block & - & - & - & - & - \\
\hline
Nakamoto Coeff & 174 & - & - & 10 & - \\
\hline
\end{tabular}
\label{tab:da_comparison}
\end{table}
