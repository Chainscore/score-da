% ================================================================
% SECTION VIII: CONCLUSION
% ================================================================

This paper presented the first comprehensive empirical benchmarking study of six production blockchain DA protocols (Polkadot ELVES, Ethereum EIP-4844/PeerDAS, Celestia, Espresso Tiramisu, NEAR, and Avail) using 90~days of mainnet data collected under a unified methodology.

\subsection{Summary of Findings}

Our analysis yields four principal findings.

\emph{First}, all six protocols operate well below their theoretical capacity ceilings ($<$50\% utilization), indicating that the current DA landscape is demand-constrained.
Observed throughput measures ecosystem adoption, not protocol engineering limits.
This observation is important for interpreting any cross-protocol throughput comparison: until protocols approach saturation, throughput comparisons reflect market dynamics rather than technical capability.

\emph{Second}, DA costs span seven orders of magnitude across protocols, from $\sim$\$3$\times$10$^{-9}$/MiB (Espresso, whose 1~wei/byte fee is a Mainnet~0 placeholder) to $\sim$\$0.80/MiB (Celestia peak during a demand spike).
Annualized cost normalization, which accounts for protocol-specific retention windows, reveals that protocols appearing cheap at spot pricing (due to short retention) may be expensive when continuous availability is required.
The absence of a single ``cheapest'' protocol after annualization underscores the importance of multi-dimensional evaluation.

\emph{Third}, no protocol implements slashing for pure data withholding.
Enforcement is uniformly indirect: Polkadot gates finality on availability bitfields, Celestia employs bad encoding fraud proofs, Ethereum uses peer scoring, Espresso relies on committee reputation, NEAR has an unused challenge mechanism, and Avail slashes only for equivocation.
This universal gap reflects the fundamental difficulty of attributing data withholding (an omission fault) to specific actors.

\emph{Fourth}, DA is universally temporary, with retention windows ranging from $\sim$85~minutes (Avail) to $\sim$30~days (Celestia).
This ephemerality is a deliberate design choice, but it creates binding constraints for downstream consumers, particularly optimistic rollups whose fraud proof windows may exceed their DA layer's retention period.

\subsection{Architectural Taxonomy}

We classified DA architectures along three paradigms (DAS, VID, and validator-set recovery) and along orthogonal dimensions of encoding scheme and commitment type.
This taxonomy reveals that the choice of verification paradigm has deeper implications than encoding or commitment choices: it determines light client capabilities, trust assumptions, and enforcement mechanisms.
DAS protocols (Celestia, Avail, Ethereum) enable trust-minimized verification but require active network participation.
Validator-recovery protocols (Polkadot, NEAR) provide deterministic guarantees but offer no independent light client verification.
VID (Espresso) provides algebraic bribery resistance at the cost of committee trust.

\subsection{Contributions and Reproducibility}

Beyond the empirical findings, this work contributes:
(i)~open-source data collection tooling for all six protocols,
(ii)~raw datasets comprising over 14~million blocks,
(iii)~analysis notebooks that reproduce all figures and tables, and
(iv)~interactive dashboards for real-time data exploration.
All artifacts are released under Apache~2.0 (code) and CC~BY~4.0 (data) licenses to enable reproducibility and longitudinal extension by the research community.

\subsection{Future Work}

Several directions emerge from this study.
\emph{Longitudinal extension}: as adoption grows and protocols approach saturation, repeating this analysis will reveal supply-constrained dynamics not observable in the current demand-constrained regime.
\emph{Stress testing}: synthetic load testing under controlled conditions would complement our observational methodology with causal measurements of protocol limits.
\emph{Withholding enforcement}: the universal absence of withholding-specific slashing identifies an open problem; formal analysis of incentive-compatible enforcement mechanisms for omission faults remains an important research direction.
\emph{Cross-layer analysis}: measuring end-to-end rollup performance (confirmation time, cost, finality) as a function of DA backend choice would directly inform the rollup developer decision framework.
\emph{PeerDAS deployment}: Ethereum's PeerDAS is in progressive deployment; measuring its impact on DA pricing and validator behavior as it reaches full activation is a natural extension of this work.
