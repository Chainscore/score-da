In this section, we detail the methods and approaches we will be using in our research. The methodology is structured as follows:

\subsection{Code Module Setup}
The code module will be set up for all the DA solution to run in a simulated environment. This will help us to benchmark the performance of each DA solution under controlled conditions. The code module will have the below specified structure:


Also the code module will be open sourced for the community to use and verify the results.
\subsection{Comparative Graphs }
We will be using various comparative factor to graph out the differences in the data availability solutions. We will also provide a code based module setup that will generate the graphs based on the performance of the actual data availability solutions.

\subsubsection{Throughput Analysis}
This section analyzes the maximum throughput capabilities of different DA solutions under various network conditions and data sizes. We measure transactions per second (TPS) and data processing capacity.

\begin{figure}[ht]
    \centering
    \includegraphics[width=0.5\textwidth]{./figures/dummy_graph.png}
    \caption{Comparative Throughput Analysis of DA Solutions}
    \label{fig:throughput}
\end{figure}

\subsubsection{Latency Measurements}
We compare the end-to-end latency of data availability confirmation across different solutions, including network propagation time and consensus finality.

\begin{figure}[ht]
    \centering
    \includegraphics[width=0.5\textwidth]{./figures/dummy_graph.png}
    \caption{Comparative latency Analysis of DA Solutions}
    \label{fig:latency}
\end{figure}


\subsubsection{Cost Analysis}
This section presents a comparative study of operational costs per megabyte of data stored and retrieved across different DA solutions.
\begin{figure}[ht]
    \centering
    \includegraphics[width=0.5\textwidth]{./figures/dummy_graph.png}
    \caption{Comparative Cost Analysis of DA Solutions}
    \label{fig:cost}
\end{figure}



\subsubsection{Scalability Metrics}
Analysis of how different DA solutions scale with increasing network size, data volume, and number of participants.

\begin{figure}[ht]
    \centering
    \includegraphics[width=0.5\textwidth]{./figures/dummy_graph.png}
    \caption{Comparative Scalability Analysis of DA Solutions}
    \label{fig:scalability}
\end{figure}



% \subsection{Data Collection}
% Data was collected through surveys distributed to a sample population. The survey included questions designed to assess various parameters relevant to our study.

% \subsection{Data Analysis}
% The collected data was analyzed using statistical methods, including descriptive statistics and inferential statistics, to draw meaningful conclusions.

% \subsection{Limitations}
% We acknowledge certain limitations in our methodology, including potential biases in survey responses and the representativeness of the sample. 

% Future research may address these limitations by employing a more diverse sample and utilizing mixed methods for data collection.