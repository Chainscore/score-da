In this section, we detail the methods and approaches we will be using in our research. The methodology is structured as follows:


\subsection{Benchmarking Design}
The benchmarking design will focus on evaluating the performance of various Data Availability (DA) solutions under controlled conditions. We will simulate different network environments and data loads to assess how each solution performs in terms of key metrics.

\subsubsection{Polkadot}

\subsubsection{Celestia}

\subsubsection{Espresso's Tiramisu}

\subsubsection{NEAR}

\subsubsection{Avail}


\subsection{Performance}
% ---- Item 1 ----
\subsubsection{Throughput Analysis}
This section will analyze the maximum throughput capabilities of different DA solutions under various network conditions/data sizes. We measure transactions per second (TPS) and data processing capacity.

\begin{itemize}
    \item \textbf{Test Parameters:} Time taken, data sizes
    \item \textbf{Metrics:} MB/s
    \item \textbf{Environment:} Simulated network with controlled conditions
\end{itemize}

\begin{figure}[h!]
    \centering
    \includegraphics[width=0.8\linewidth]{./figures/dummy_graph.png}
    \caption{Comparative Throughput Analysis of DA Solutions}
    \label{fig:throughput1}
\end{figure}

\vspace{1em}

% ---- Item 3 ----
\subsubsection{TPS}
This section analyzes the transactions per second (TPS) capabilities of different DA solutions. We measure transactions per second (TPS) and data processing capacity.
\begin{itemize}
    \item \textbf{Test Parameters:} Network conditions, data sizes, node count
    \item \textbf{Metrics:} TPS
    \item \textbf{Environment:} Simulated network with controlled conditions
\end{itemize}

\begin{figure}[h!]
    \centering
    \includegraphics[width=0.8\linewidth]{./figures/dummy_graph.png}
    \caption{Comparative Latency Analysis of DA Solutions}
    \label{fig:latency1}
\end{figure}

\vspace{1em}

% ---- Item 5 ----
\subsubsection{Cost per MB}
This section analyzes the cost per megabyte (MB) of data processed by different DA solutions. We measure the cost efficiency and resource utilization.
\begin{itemize}
    \item \textbf{Test Parameters:} Data sizes, Cost in native token conversion to USD
    \item \textbf{Metrics:} \$/MB
    \item \textbf{Environment:} Simulated network with controlled conditions
\end{itemize}

\begin{figure}[h!]
    \centering
    \includegraphics[width=0.8\linewidth]{./figures/dummy_graph.png}
    \caption{Comparative Cost per MB Analysis of DA Solutions}
    \label{fig:costpermb}
\end{figure}

\vspace{1em}


\subsubsection{Latency}
This section will analyze the maximum time taken by the DA solution to finalize a block.

\begin{itemize}
    \item \textbf{Test Parameters:} Network conditions, data sizes, node count
    \item \textbf{Metrics:} seconds (s)
    \item \textbf{Environment:} Simulated network with controlled conditions
\end{itemize}

\begin{figure}[h!]
    \centering
    \includegraphics[width=0.8\linewidth]{./figures/dummy_graph.png}
    \caption{Comparative Latency Analysis of DA Solutions}
    \label{fig:latency1}
\end{figure}

\vspace{10em}



% ---- Item 4 ----
\subsubsection{Max Block Size}
This section analyzes the maximum block size capabilities of different DA solutions. We measure the maximum block size and data processing capacity.
\begin{itemize}
    \item \textbf{Test Parameters:} Network conditions, data sizes, node count
    \item \textbf{Metrics:} Block size (MB)
    \item \textbf{Environment:} Simulated network with controlled conditions
\end{itemize}

\begin{figure}[h!]
    \centering
    \includegraphics[width=0.8\linewidth]{./figures/dummy_graph.png}
    \caption{Comparative Block Size Analysis of DA Solutions}
    \label{fig:blocksize}
\end{figure}

% remove this vspace later
\vspace{1em}


\subsection{Security Assumptions}

% change thisss later
In this section, we outline the security assumptions made during the evaluation of the DA solutions. These assumptions are critical for understanding the context in which the solutions are being assessed and the potential vulnerabilities that may arise.


\subsection{Validator Costing}
In this section, we will analyze the costs incurred by validators in maintaining and operating each DA solution. This includes hardware costs, bandwidth costs, and any other operational expenses associated with running a validator node.


\subsection{Worst Case Performance}
In this section, we will evaluate the worst-case performance scenarios for each DA solution. This includes analyzing how each solution handles extreme conditions such as high network latency, large data loads, and potential attacks or failures within the network.

% \subsection{Parameters descriptions}


% We will be using the following parameters for this comparative analysis. Having a general understanding of these parameters will help in better understanding the study.

% \vspace{1em}

% 1. \textbf{Consensus Mechanism}: The underlying protocol on chain that ensures the agreement on the state of the blockchain among validator nodes.For example, Proof of Stake (PoS), Byzantine Fault Tolerance (BFT), Proof of Work (PoW) etc.
% \vspace{1em}

% 2. \textbf{Data Availability Model}: The specific approach used to ensure that all necessary data is accessible and retrievable by all participants in the blockchain network. For example, Reed-Solomon Erasure Coding, 2D Reed-Solomon with Namespaced Merkle Trees etc.
% \vspace{1em}

% 3. \textbf{Throughput}: The maximum rate at which a DA solution can process data, typically measured in Megabytes per second (MB/s).
% \vspace{1em}

% 4. \textbf{Latency}: The time taken for a DA solution to finalize a block, typically measured in seconds (s).
% \vspace{1em}

% 5. \textbf{Transactions Per Second (TPS)}: The number of transactions a DA solution can handle per second.
% \vspace{1em}

% 6. \textbf{Block Size}: The maximum size of a block that can be processed by the DA solution, typically measured in Megabytes (MB).
% \vspace{1em}

% 7. \textbf{Cost per MB}: The cost incurred to process one Megabyte (MB) of data using the DA solution, typically measured in USD.
% \vspace{1em}

% 8. \textbf{Nakamoto Coefficient}: A measure of the decentralization of a blockchain network, indicating the minimum number of entities that would need to collude to compromise the network's security.
% \vspace{1em}

% \subsection{Code Module Setup}
% The code module will be set up for all the DA solution to run in a simulated environment. This will help us to benchmark the performance of each DA solution under controlled conditions. The code module will have the below specified structure:


% Also the code module will be open sourced for the community to use and verify the results.
% \subsection{Comparative Graphs}
% We will be using various comparative factor to graph out the differences in the data availability solutions. We will also provide a code based module setup that will generate the graphs based on the performance of the actual data availability solutions.


% ---- Item 2 ----





% \subsection{Data Collection}
% Data was collected through surveys distributed to a sample population. The survey included questions designed to assess various parameters relevant to our study.

% \subsection{Data Analysis}
% The collected data was analyzed using statistical methods, including descriptive statistics and inferential statistics, to draw meaningful conclusions.

% \subsection{Limitations}
% We acknowledge certain limitations in our methodology, including potential biases in survey responses and the representativeness of the sample. 

% Future research may address these limitations by employing a more diverse sample and utilizing mixed methods for data collection.