% ================================================================
% SECTION VI: SECURITY AND TRUST ANALYSIS
% ================================================================

This section analyzes the security properties and trust assumptions of each DA protocol, focusing on Byzantine fault tolerance, data withholding enforcement, retention guarantees, and decentralization characteristics.

\subsection{Byzantine Fault Tolerance}
\label{sec:security-bft}

All six protocols inherit the classical BFT bound of $f < n/3$, where $f$ is the number of Byzantine-faulty nodes and $n$ is the total committee or validator set size.
However, the practical manifestation of this threshold varies substantially.

\paragraph{Polkadot}
Polkadot's relay chain requires $\geq 2/3$ of validators to sign availability bitfields before a parachain block can be included and finalized.
With approximately 600~active validators, the system tolerates up to $\sim$200 Byzantine faults.
The ELVES protocol further strengthens this through approval checking: randomly selected validators re-execute parachain blocks post-inclusion, and a single invalid execution triggers a dispute that can slash the responsible validators~\cite{Jeff_Burdges_ELVES}.
The probability of an invalid block escaping detection decreases exponentially with the number of approval checkers.

\paragraph{Ethereum}
Ethereum's Casper FFG requires a $2/3$ supermajority of validators (by stake weight) to finalize blocks.
With approximately 1~million active validators, the economic security backing finality exceeds \$30~billion (at ETH $\sim$\$3,000).
PeerDAS distributes custody across the validator set: each validator holds a subset of blob columns, and system-level availability confidence emerges from the aggregate custody distribution~\cite{EIP-7594}.
The large validator set makes coordinated withholding attacks economically prohibitive.

\paragraph{Celestia}
Celestia runs CometBFT (Tendermint) consensus, which requires $\geq 2/3$ of validators (by stake weight) to commit each block.
The validator set is capped at 100 in the current configuration.
DAS provides an additional layer of security: even if validators finalize a block with unavailable data, light client sampling will probabilistically detect the withholding~\cite{AlBassam_FraudDA}.
This two-layer model (consensus BFT + sampling) provides stronger guarantees than consensus alone, at the cost of requiring sufficient light client participation.

\paragraph{Espresso}
Espresso's HotShot consensus operates with 100~nodes across 20~operators in the Mainnet~0 configuration.
The DA certificate requires $\geq 2/3$ of committee signatures, tolerating up to $\sim$33 Byzantine nodes.
However, the small operator count (20) means that the effective trust assumption is closer to $f < 7$ at the operator level~\cite{Espresso:Hotshot_and_Triamisu}.

\paragraph{NEAR}
NEAR's Nightshade sharding applies the BFT threshold per shard.
With 9~shards and approximately 400~validators, the per-shard committee size depends on stake distribution.
A critical assumption is that each shard independently maintains $< 1/3$ Byzantine nodes; an adversary who concentrates stake in a single shard can potentially compromise that shard's DA at lower total cost than compromising the network-wide threshold~\cite{NEAR_Nightshade_Scaling_Blockchain}.
In practice, however, most mainnet validators currently track all shards rather than only their assigned shard, which means the per-shard BFT assumption is more theoretical than operational: DA benefits from full-network replication even though the protocol is designed for shard-local storage.

\paragraph{Avail}
Avail's BABE/GRANDPA consensus requires $\geq 2/3$ honest validators for both block production and finality.
DAS light clients provide independent verification, analogous to Celestia's model.
The VectorX bridge to Ethereum introduces an additional trust surface: bridge liveness depends on the prover infrastructure, and the approximately 2-hour bridge delay creates a window during which Avail-native finality and Ethereum-bridged attestations may diverge~\cite{Avail_The_DA_Blockchain}.

Table~\ref{tab:bft_comparison} summarizes the BFT characteristics across protocols.

\begin{table*}[t]
\centering
\caption{Byzantine Fault Tolerance and Validator Set Characteristics}
\label{tab:bft_comparison}
\setlength{\tabcolsep}{4pt}
\renewcommand{\arraystretch}{1.2}
\begin{tabular}{lccccc}
\toprule
\textbf{Protocol} & \textbf{BFT Threshold} & \textbf{Validator Count} & \textbf{Consensus} & \textbf{Finality Latency} & \textbf{Staking Requirement} \\
\midrule
Polkadot & $f < n/3$ & $\sim$600 & BABE + GRANDPA & 12--60\,s & $\sim$1.8M DOT min \\
Ethereum & $f < n/3$ (stake) & $\sim$1M & Casper FFG & $\sim$12.8\,min & 32 ETH per validator \\
Celestia & $f < n/3$ (stake) & $\sim$100 & CometBFT & Single-slot ($\sim$6\,s) & Variable \\
Espresso & $f < n/3$ & 100 (20 operators) & HotShot & $\sim$1--2\,s & None (permissioned) \\
NEAR & $f < n/3$ per shard & $\sim$400 & Nightshade & $\sim$2\,s & Seat-price auction \\
Avail & $f < n/3$ & $\sim$100 & BABE + GRANDPA & 20--60\,s & Variable \\
\bottomrule
\end{tabular}
\end{table*}

\subsection{Data Withholding Enforcement}
\label{sec:security-withholding}

As established in Section~\ref{subsec:verification}, no protocol implements slashing specifically for data withholding.
Here we analyze the indirect enforcement mechanisms and their practical effectiveness.

\paragraph{Detection mechanisms}
Protocols employ three distinct detection approaches:

\begin{enumerate}
    \item \textbf{Sampling-based detection} (Celestia, Avail, Ethereum/PeerDAS): Light clients randomly request coded pieces; failure to receive responses indicates potential withholding. Detection probability increases with sampling participation. This approach requires an active network of sampling nodes and is vulnerable to targeted eclipsing attacks.

    \item \textbf{Consensus-integrated attestation} (Polkadot, NEAR): Validators attest to chunk availability through protocol-defined messages (Polkadot's availability bitfields, NEAR's chunk endorsements). Detection is deterministic---a missing attestation is visible on-chain---but only detects validator-level withholding, not targeted withholding from specific requesters.

    \item \textbf{Committee certification} (Espresso): The DA certificate requires threshold signatures from committee members who have verified receipt of their VID shares. A missing certificate prevents finalization. Detection is deterministic but relies entirely on the committee's honest participation.
\end{enumerate}

\paragraph{Enforcement gap}
The fundamental challenge across all protocols is the \emph{attribution gap}: detecting that data is unavailable does not identify the responsible party.
A validator that fails to serve a chunk may be offline, network-partitioned, or maliciously withholding, and the protocol cannot distinguish these cases.
This attribution difficulty explains why no protocol has implemented withholding-specific slashing: the false positive rate (slashing honest nodes experiencing network issues) would be unacceptable.

\subsection{Retention and Persistence}
\label{sec:security-retention}

DA is universally treated as temporary across all six protocols.
Table~\ref{tab:retention} summarizes retention windows and the implications for downstream consumers.

\begin{table}[t]
\centering
\caption{Data Retention Windows and Persistence Model}
\label{tab:retention}
\setlength{\tabcolsep}{4pt}
\renewcommand{\arraystretch}{1.2}
\begin{tabular}{lcc}
\toprule
\textbf{Protocol} & \textbf{Retention Window} & \textbf{Pruning Mechanism} \\
\midrule
Avail & $\sim$85\,min (256 blocks) & Substrate state pruning \\
Polkadot & $\sim$25\,h & Relay chain erasure pruning \\
NEAR & $\sim$2.5\,d (5 epochs) & Garbage collection \\
Espresso & 1--7\,d (configurable) & Query service retention \\
Ethereum & $\sim$18\,d (4{,}096 epochs) & Consensus-layer pruning \\
Celestia & $\sim$30\,d & Node storage rotation \\
\bottomrule
\end{tabular}
\end{table}

The retention range spans two orders of magnitude, from Avail's $\sim$85~minutes to Celestia's $\sim$30~days.
This variation has critical implications for rollup security:

\begin{itemize}
    \item \textbf{Fraud proof windows}: Optimistic rollups require a challenge period (typically 7~days) during which fraud proofs can be submitted; the DA layer must retain data at least as long as the challenge period. Avail (85~min), Polkadot (25~h), and NEAR (2.5~d) all have retention windows shorter than the standard 7-day fraud proof window, requiring rollups to either archive data externally or implement shorter challenge periods.

    \item \textbf{Bridging latency}: Cross-chain bridges that verify DA attestations must complete verification within the retention window. Avail's VectorX bridge to Ethereum has an approximately 2-hour attestation delay, consuming a significant fraction of the 85-minute retention window (with race conditions in edge cases).

    \item \textbf{Archival burden}: After DA expiry, data persistence shifts to voluntary archival by full nodes, third-party indexers, or rollup operators themselves. No protocol guarantees post-retention data retrievability.
\end{itemize}

\subsection{Trust Assumptions Summary}
\label{sec:security-trust}

Table~\ref{tab:trust_summary} maps each protocol's key trust assumptions.

\begin{table*}[t]
\centering
\caption{Trust Assumption Comparison Across DA Protocols}
\label{tab:trust_summary}
\setlength{\tabcolsep}{4pt}
\renewcommand{\arraystretch}{1.2}
\small
\begin{tabular}{lp{4cm}p{4cm}p{4cm}}
\toprule
\textbf{Protocol} & \textbf{Availability Assumption} & \textbf{Encoding Trust} & \textbf{Additional Assumptions} \\
\midrule
Polkadot & $\geq 2/3$ validators honest (bitfield voting) & Validators re-encode and verify (approval checking) & Relay chain liveness \\
Ethereum & $\geq 2/3$ validators honest (Casper FFG) + sufficient custody distribution & KZG algebraic verification (no fraud proofs needed) & Trusted setup integrity (KZG ceremony) \\
Celestia & $\geq 2/3$ validators honest + sufficient DAS participation & BEFPs from $\geq 1$ honest full node & $\geq 1$ non-eclipsed honest full node \\
Espresso & $\geq 2/3$ committee signs DA certificate & KZG + vector commitment algebraic verification & Committee honesty; Ladyfinger fallback liveness \\
NEAR & $< 1/3$ Byzantine per shard & Chunk producers re-encode & Honest shard assignment; no stake concentration \\
Avail & $\geq 2/3$ validators honest + DAS participation & KZG algebraic verification & Trusted setup integrity; VectorX bridge liveness \\
\bottomrule
\end{tabular}
\end{table*}

The trust models fall into three categories.
\emph{DAS-enhanced protocols} (Celestia, Avail, Ethereum) provide the strongest light client guarantees but require active network participation in sampling.
\emph{Validator-recovery protocols} (Polkadot, NEAR) provide deterministic guarantees conditioned on validator honesty but offer no independent light client verification.
\emph{VID-based protocols} (Espresso) provide algebraic guarantees against bribery but depend on committee and fallback infrastructure availability.
